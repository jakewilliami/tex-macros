\documentclass{beamer}
\usepackage[utf8]{inputenc}
\usepackage[T1]{fontenc}
\title{Divisibility}
\date[Math 161]{Math 161 -- Discrete Mathematics and Logic}
\author[Dr Day]{Dr Adam Day}

\newcommand{\Z}{\mathbb{Z}}
\newcommand{\N}{\mathbb{N}}

\usetheme{vuw}

\begin{document}


\begin{frame}
\titlepage
\end{frame}


\begin{frame}
 \frametitle{The Divisibility Relation}

\begin{itemize}
 \item $\N = \{0,1,2,3,4,5,\ldots\}$
\item $\Z =\{\ldots, -3,-2,-1,0,1,2,3,\ldots\}$
\end{itemize}

\begin{definition}
 Let $a$ and $b$ be integers. We say that $a$ \textit{divides} $b$ and  write $a \mid b$ if for some integer $e$ we have that $a\cdot e= b$. 
\end{definition}

\begin{itemize}
 \item if $a$ does not divide $b$ we write $a \nmid b$. 
 \item e.g.\ $4 \mid 12$, $3 \nmid 100$.
 \item For any $n$, $n \mid 0$. 
 \end{itemize}


\begin{lemma}
The divisibility relation on $\Z$ is reflexive and transitive.
\end{lemma}

\end{frame}



\begin{frame}
\frametitle{More Lemmas}

\begin{definition}
 Let $a$ and $b$ be integers. We say that $a$ \textit{divides} $b$ and  write $a \mid b$ if for some integer $e$ we have that $a\cdot e= b$. 
\end{definition}

\vfill

 \begin{lemma}
 If $a$ and $b$ are integers such that $a \mid b$ and $b \mid a$ then $a= \pm b$. 
\end{lemma}

\vfill



\begin{lemma}
\label{lem: sum}
Let $a,b,c \in \Z$. 
 If $a\mid b$ and $a \mid c$ then $a \mid (b +c)$ and $a\mid (b-c)$.
\end{lemma}
\end{frame}



\end{document}


